\documentclass{article}

\usepackage[param=ok]{graphicx}
\usepackage{amsmath}
\usepackage{tikz}

\title{LaTeX Example}
\author{Author Name}
\date{\today}

\begin{document}

\maketitle

\chapter{bbbbbbbb}

\chapter*{aaaaa}
\section{Introduction}

This is a text paragraph. You can include \textit{italic text}, \textbf{bold text}, and \texttt{monospaced text}. $y = \vec a x + b - \odot b$

\section{Lists}

\begin{itemize}
    \item Bullet list item 1
    \item Bullet list item 2
\end{itemize}

\begin{enumerate}
    \item Numbered list item 1
    \item Numbered list item 2
\end{enumerate}

\section{Table}

\begin{tabular}{|c|c|}
    \hline
    Column 1 & Column 2 \\
    \hline
    Row 1 & 1 \\
    Row 2 & 2 \\
    \hline
\end{tabular}

\section{Figure}

\begin{figure}[h]
    \centering
    \includegraphics[width=0.5\textwidth]{example.png}
    \caption{Example figure}
\end{figure}

\section{Math}

Here's a mathematical equation:

\begin{equation}
    y = mx + b
\end{equation}

\begin{comment}
    ondoinsefos
    pojsefoisf
    fsonf
\end{comment}

$$
A =
\begin{bmatrix}
a & b \\
c & d \\
\end{bmatrix}
$$

% Define a new command \hello that prints "Hello, World!"
\newcommand{\hello}{Hello, World!}

% Use the new command
\hello

% Redefine (renew) the command to print "Hello, LaTeX!"
\renewcommand{\hello}{Hello, LaTeX!}

% Use the renewed command
\hello

\input{/some/path/to/file.tex}
\include{/some/path/to/file.tex}

\begin{tikzpicture}
    % Nodes
    \node[draw, circle] (A) at (0,0) {A};
    \node[draw, rectangle] (B) at (2,0) {B};
    \node[draw, diamond, fill=yellow] (C) at (1,-2) {C};

    % Edges
    \draw[->] (A) -- (B);
    \draw[->] (B) -- (C);
    \draw[->] (C) -- (A);

    % Loop
    \draw[->] (A) edge[loop above] node {} (A);

    % Grid
    \draw[step=1cm,gray,very thin] (-1.9,-2.9) grid (2.9,0.9);
\end{tikzpicture}


\end{document}